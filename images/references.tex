\subsection*{Публикации по теме диссертационной работы}

\subsubsection*{В рецензируемых научных изданиях, входящих в перечень Высшей аттестационной комиссии при Министерстве образования и науки Российской Федерации и в международную реферативную базу данных Scopus}%перечень рецензируемых научных изданий ВАК} % ВАК и индексируемых Scopus

% \noindent~~~2.~~Петухов Д. С. Моделирование и управление расходно-напорными характеристиками имплантируемого насоса крови АВК-Н <<Спутник>> // Медицинская техника. 2016. № 6. С. 52--55.

\renewcommand{\baselinestretch}{1.5}
\renewcommand{\arraystretch}{1.5}
\definecolor{bib3}{HTML}{767676} % 575757

\begin{enumerate}[leftmargin=25pt, itemsep=12pt] %[leftmargin=2em, itemsep=2em]
  \item Петухов Д. С., Телышев Д. В. Исследование роторного насоса для поддержки кровообращения правого желудочка сердца при механической поддержке кровообращения обоих желудочков сердца // Медицинская техника. 2017. № 1. С. 24--26. Версия Scopus: \textcolor{bib3}{Petukhov D. S., Telyshev D. V. Performance of a right ventricular assist rotary pump in the process of biventricular mechanical circulatory support // Biomedical Engineering. 2017. Vol. 51, no. 1. P. 33--36.}
  \item Петухов Д. С. Моделирование и управление расходно-напорными характеристиками имплантируемого насоса крови АВК-Н <<Спутник>> // Медицинская техника. 2016. № 6. С. 52--55. Версия Scopus: \textcolor{bib3}{Petukhov, D. S. Simulation and control of the H--Q curves of the Sputnik portable ventricular assist device // Biomedical Engineering. 2017. Vol. 50, no. 6. P. 429--432.}
  \item Петухов Д. С., Телышев Д. В. Математическая модель сердечно-сосудистой системы педиатрических пациентов с врожденными пороками сердца // Медицинская техника. 2016. № 4. С. 9--11. Версия Scopus: \textcolor{bib3}{Petukhov D. S., Telyshev D. V. A mathematical model of the cardiovascular system of pediatric patients with congenital heart defect // Biomedical Engineering. 2016. Vol. 50, no. 4. P. 229--232.}
  \item Петухов Д. С., Телышев Д. В. Алгоритмы управления роторными аппаратами вспомогательного кровообращения // Медицинская техника. 2016. № 3. С. 8--11. Версия Scopus: \textcolor{bib3}{Petukhov D. S., Telyshev D. V. Control algorithms for rotary blood pumps used in assisted circulation // Biomedical Engineering. 2016. Vol. 50, no. 3. P. 157--160.}
  \item Петухов Д. С., Телышев Д. В., Селищев С. В. Метод управления роторным насосом крови для системы вспомогательного кровообращения левого желудочка сердца // Современные технологии в медицине. 2016. Т. 8, № 1. С. 28--33. Версия Scopus: \textcolor{bib3}{Petukhov D. S., Telyshev D. V., Selishchev S. V. Control method of a rotary blood pump for a left ventricular assist device. Sovremennye tehnologii v medicine 2016. Vol. 8, no. 1. P. 28--33.}
  \item Петухов Д. С., Телышев Д. В. Исследование чувствительности роторного насоса крови <<Спутник>> к преднагрузке и постнагрузке // Медицинская техника. 2015. № 6. С. 27--30. Версия Scopus: \textcolor{bib3}{Petukhov D. S., Telyshev D. V., Analysis of the preload and afterload sensitivity of the Sputnik rotary blood pump // Biomedical Engineering. 2016. Vol. 49, no. 6. P. 362--365.}
  \item Петухов Д. С., Селищев С. В., Телышев Д. В. Перспективы развития технологий полной замены функции сердца с помощью механических систем поддержки кровообращения // Медицинская техника. 2015. № 5. С. 5--8. Версия Scopus: \textcolor{bib3}{Petukhov D. S., Selishchev S. V., Telyshev D. V. Prospects for development of technologies for complete replacement of heart function by mechanical circulatory support systems // Biomedical Engineering. 2016. Vol. 49, no. 5. P. 258--262.}
  \item Петухов Д. С., Селищев С. В., Телышев Д. В. Полностью искусственное сердце: современное состояние // Медицинская техника. 2015. № 4. С. 1--4. Версия Scopus: \textcolor{bib3}{Petukhov D. S., Selishchev S. V., Telyshev D. V. Total artificial heart: state-of-the-art // Biomedical Engineering. 2015. Vol. 49, no. 4. P. 193--196.}
  \item Петухов Д. С., Телышев Д. В. Моделирование изменений в динамике течения крови через имплантируемый осевой насос // Медицинская техника. 2014. № 6. С. 44--47. Версия Scopus: \textcolor{bib3}{Petukhov D. S., Telyshev D. V. Simulation of blood flow dynamics changes through implantable axial flow pump // Biomedical Engineering. 2015. Vol. 48, no. 6. P. 336--340.}
  \item Петухов Д. С., Селищев С. В., Телышев Д. В. Развитие аппаратов вспомогательного кровообращения левого желудочка сердца как наиболее эффективный способ лечения острой сердечной недостаточности // Медицинская техника. 2014. № 4. С. 37--39. Версия Scopus: \textcolor{bib3}{Petukhov D. S., Selishchev S. V., Telyshev D. V. Development of left ventricular assist devices as the most effective acute heart failure therapy // Biomedical Engineering. 2015. Vol. 48, no. 6. P. 328--330.}
  \item Петухов Д. С., Селищев С. В. Оценка изменений в работе правого желудочка сердца при наличии аппарата вспомогательного кровообращения левого желудочка сердца // Медицинская техника. 2014. № 4. С. 28--32. Версия Scopus: \textcolor{bib3}{Petukhov D. S., Selishchev S. V. Assessment of changes in right ventricle function in patients with left ventricular assist device // Biomedical Engineering. 2014. Vol. 48, no. 4. P. 204--208.}
\end{enumerate}

\subsubsection*{В тезисах докладов всероссийских и международных конференций}

\renewcommand{\baselinestretch}{1.5}
\renewcommand{\arraystretch}{1.5}

\begin{enumerate}[leftmargin=25pt, itemsep=12pt]  %[leftmargin=2em, itemsep=2em]
  \item Petukhov D. S. A control algorithm of flow balance for a biventricular assist device // 2nd International Symposium <<Physics, Engineering and Technologies for Biomedicine>>. 2017. P. 341--342.
  \item Petukhov D. S. Quantitative assessment of heart-pump interaction for an axial-flow rotary blood pump Sputnik: in vitro study // 44th ESAO and 7th IFAO Congress. 2017. P. 454.
  \item Petukhov D. S., Telyshev D. V. An approach to the evaluation and control of a rotary blood pump using in vitro experimental results for two generations of LVAD Sputnik // 24th Congress of the International Society for Rotary Blood Pumps. 2016. P. 70.
  \item Petukhov D. S. Development of a cardiovascular system model for investigation of biventricular circulatory support // XII Russian-German Conference on Biomedical Engineering. 2016. P. 200--204.
  \item Petukhov D. S., Telyshev D. V., Selishchev S. V. A method for identification of pumping states in an implantable rotary blood pump: experimental validation for the LVAD Sputnik // 62nd ASAIO Annual Conference. 2016. P. 10.
  \item Петухов Д. С. Концепция метода управления роторным насосом крови путем определения режимов работы насоса на основе результатов in vitro испытаний АВК <<Спутник>> // 23-я Всероссийская конференция <<Микроэлектроника и информатика>>. 2016. С. 272.
  \item Petukhov D. S., Telyshev D. V. Investigation of control objectives for the heart failure treatment using the control strategy of a rotary blood pump // 42th ESAO conference. 2015. P. 403.
  \item Petukhov D. S., Telyshev D. V. Comparative study of influence of two rotary blood pumps on the cardiovascular system // 37th Annual International Conference of the IEEE Engineering in Medicine and Biology Society. 2015. P. 115.
  \item Petukhov D. S., Telyshev D. V. Design concept of patient-adaptive control method for a ventricular assist device // 37th Annual International Conference of the IEEE Engineering in Medicine and Biology Society. 2015. P. 116.
  \item Petukhov D. S., Telyshev D. V. Control strategy for an implantable rotary blood pump based on identification of pumping states // 61st ASAIO Annual Conference. 2015. P. 4.
  \item Petukhov D. S., Telyshev D. V. A method for identification of pumping states of an implantable rotary blood pump // XI German-Russian Conference on Biomedical Engineering. 2015. P. 159--161.
  \item Петухов Д. С. Определение основных требований к системе управления аппаратом вспомогательного кровообращения левого желудочка сердца // 22-я Всероссийская конференция <<Микроэлектроника и информатика>>. 2015. С. 325.
  \item Петухов Д. С., Селищев С. В., Телышев Д. В. Критерии неинвазивной оценки расхода имплантируемого осевого насоса крови // 16-я научно-техническая конференция <<МедТех>>. 2014. С. 162--163.
  \item Петухов Д. С. Анализ изменений в гемодинамике для случая механической поддержки кровообращения обоих желудочков сердца // 11-я международная конференция <<Физика и радиоэлектроника в медицине и экологии>>. 2014. С. 213--214.
  \item Petukhov D. S. Simulation of hemodynamic changes associated with the right ventricular failure in the presence of a left ventricular assist device // X Russian-German Conference on Biomedical Engineering. 2014. P. 122--123.
  \item Петухов Д. С. Моделирование эффекта гистерезиса в расходных характеристиках имплантируемого осевого насоса крови // 6-я Троицкая конференция <<Медицинская физика и инновации в медицине>>. 2014. С. 530--531.
  \item Петухов Д. С. Разработка математической модели сердечно-сосудистой системы для изучения изменений в гемодинамике при поддержке аппарата вспомогательного кровообращения // 21-я Всероссийская конференция <<Микроэлектроника и информатика>>. 2014. С. 227.
  \item Петухов Д. С. Моделирование популяционной динамики красных кровяных телец и ее нарушения в случае гипохромных анемий // 20-я Всероссийская конференция <<Микроэлектроника и информатика>>. 2013. С. 284.
\end{enumerate}

\subsection*{Список цитируемой литературы}

\begin{enumerate}[leftmargin=25pt, itemsep=12pt]
 \item Гроп Д. Методы идентификации систем. Мир, 1979.
 \item Льюнг Л. Идентификация систем. Наука, 1991.
 \item Акулов С. А., Федотов А. А. Основы теории биотехнических систем. Физматлит, 2014.
 \item Трояновский В. М. Компьютерное моделирование процедур идентификации динамических объектов // Известия высших учебных заведений. Электроника. 2008. № 4. С. 16--17.
% \item Трояновский В. М. Анализ и параметрический синтез стохастических систем управления: диссертация доктора технических наук: 05.13.01 / Технологический институт Южного федерального университета, Таганрог. 2008. С. 285.
 \item Иткин Г. П., Филатов И. А., Дозоров К. Н. Косвенные методы определения расхода и напора роторных насосов для крови // Вестник трансплантологии и искусственных органов. 2015. Т. 17. № 1. С. 97--102.
 \item Дозоров К. Н. Биотехническая система мониторинга и управления вспомогательным роторным насосом крови: автореферат диссертации кандидата технических наук: 05.11.17 / МГТУ им. Н. Э. Баумана, Москва. 2009. С. 16.
 \item Солодянников Ю. В. Элементы математического моделирования и идентификация системы кровообращения: монография // Издательство Самарского университета. 1994.
 \item Тмур А. Б. Методы идентификации технологического процесса трубопроводного транспорта нефти: автореферат диссертации кандидата технических наук: 05.13.06 / Институт проблем управления им. В.А. Трапезникова Российской академии наук, Москва. 2014. С. 23.
 \item Moscato F., Danieli G. A., Schima H. Dynamic modeling and identification of an axial flow ventricular assist device // The International journal of artificial organs. 2009. Vol. 32, no. 6. P. 336--343.
 \item Pirbodaghi T. Mathematical modeling of rotary blood pumps in a pulsatile in vitro flow environment // Artificial organs. 2017. Vol. 41, no. 8. P. 710-716.
 \item Bovendeerd P. H., Borsje P., Arts T., Vosse F. N. Dependence of intramyocardial pressure and coronary flow on ventricular loading and contractility: a model study // Annals of Biomedical Engineering. 2006. Vol. 34, no. 12. P. 1833--1845.
 \item Martina J. R., Bovendeerd P. H., de Jonge N. et al. Simulation of changes in myocardial tissue properties during left ventricular assistance with a rotary blood pump // Artificial Organs. 2013. Vol. 37, no. 6. P. 531--540.
 \item Misgeld B. J., R\"uschen D., Schwandtner S. et al. Robust decentralised control of a hydrodynamic human circulatory system simulator // Biomedical Signal Processing and Control. 2015. Vol. 20, no. 1. P. 35--44.
  \item Price K., Storn R. M., Lampinen J. A. Differential evolution: a practical approach to global optimization. Springer Science \& Business Media, 2006.

\end{enumerate}

%\newpage
~
\vspace{500px}
~\newpage
\vspace{0px}~\vskip20.7cm \thispagestyle{empty}
\begin{center}
\rule{0.80\textwidth}{.5pt}

Подписано в печать:

Заказ № 23 Тираж 100 экз. Уч.-изд.л. 1,5 Формат 60$\times$84 1/16 % 23/20 - 1.15 в большую сторону 1,2 % 25/20 -- 1.25 в большую сторону 1,3

Отпечатано в типографии ИПК МИЭТ.

124498, Москва, Зеленоград, площадь Шокина, д. 1, МИЭТ.

\rule{0.80\textwidth}{.5pt}
\end{center}
